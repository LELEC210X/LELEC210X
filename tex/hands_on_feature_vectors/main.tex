\documentclass[a4paper,11pt]{article}
\usepackage[utf8]{inputenc}
\usepackage[T1]{fontenc}
\usepackage{amsmath}
\usepackage{mathtools}
\usepackage{amsfonts}
\usepackage{amssymb}
\usepackage{graphicx}
\usepackage{multicol}
\usepackage{array}
\usepackage{float}
\usepackage{epstopdf}
\usepackage{caption}
\usepackage{subcaption}
\usepackage{gensymb}
\usepackage[bottom]{footmisc}
\usepackage{appendix}
\usepackage{pdfpages}
\usepackage{todonotes}
\usepackage{mathpazo}
\usepackage{titleps}
\usepackage{color}
\usepackage{xcolor}
\usepackage{colortbl}
\usepackage{siunitx}
\usepackage{pdflscape}
\usepackage{cancel}

\usepackage[skins]{tcolorbox}
\usepackage{sectsty}
\usepackage[arrowmos]{circuitikz}
\usepackage{pgfplots}
\usepackage{blindtext}
\usepackage[inner=2cm,outer=2cm,top=2.5cm,bottom=2.5cm]{geometry}
\usepackage{todonotes}
\usepackage{url}
\usepackage{adjustbox}
\usepackage{tabularx}
\usepackage{booktabs}
\usepackage{fancybox}
\usepackage[tikz]{bclogo}

%%%%%%%%%%%%%%% Make references colors %%%%%%%%%%%%%%%%%%
\usepackage{hyperref}
\hypersetup{
    colorlinks=true,
    linkcolor=black,
    filecolor=magenta,
    urlcolor=blue,
    citecolor = magenta
}
\usepackage{cleveref}
\crefformat{equation}{(#2\color{blue}#1#3)}
%%%%%%%%%%%%%%%%%%%%%%%%%%%%%%%%%%%%%%%%%%%%%%%%%%%%%%%%


%For code insertion
\usepackage{listings}
\lstdefinestyle{customc}{
  belowcaptionskip=1\baselineskip,
  breaklines=true,
  frame=L,
  xleftmargin=\parindent,
  language=C,
  showstringspaces=false,
  basicstyle=\footnotesize\ttfamily,
  keywordstyle=\bfseries\color{green!40!black},
  commentstyle=\itshape\color{purple!40!black},
  identifierstyle=\color{blue},
  stringstyle=\color{orange},
}

\lstdefinestyle{customasm}{
  belowcaptionskip=1\baselineskip,
  frame=L,
  xleftmargin=\parindent,
  language=[x86masm]Assembler,
  basicstyle=\footnotesize\ttfamily,
  commentstyle=\itshape\color{purple!40!black},
}

\lstset{escapechar=@,style=customc}
%End for code insertion


\graphicspath{{figures/}}
\sectionfont{\large}
\subsectionfont{\normalsize}

%%%%%%%%%%%%%%%%%%%
% HANDS-ON NUMBER
\newcommand\handsOn{MCU}
%%%%%%%%%%%%%%%%%%%

\newpagestyle{main}{
	\sethead[LELEC2102: Hands-on \handsOn][][]{LELEC2102: Hands-on \handsOn}{}{}
	\headrule
    \setfoot[][\thepage][]{}{\thepage}{}
}

\newcommand{\horrule}[1]{\rule{\linewidth}{#1}} % Create horizontal rule command with 1 argument of height

%%%%%%%%%%%%%%%%%%%%%%%%%%%%%%%%%%%%%%%%%%%%%%%%%%%%%%%%%%%%%%%%%%%%%%%%%%%%

\begin{document}
\renewcommand{\figurename}{Fig.}

\renewcommand{\thepage}{\arabic{page}}
\setcounter{page}{1}
\pagestyle{main}
\newpage \clearpage

\begin{center}
\begin{huge}
Hands-On \handsOn: Embedded feature vector computation\\
\end{huge}
\vspace{0.3cm}
%\textit{TA 1, TA 2}
\end{center}

% TO CHANGE:
% Stop_cycle_count <- read_cycle_count

% Perspectives for next year:
% - add count_cycles in the project et leurs faire comprendre quelles fonctions utilisent beaucoup + de cycles.
% (comme ça on leurs fera comprendre, en + de la notion théorique de complexité, le fait qu'en embarqué c'est le nombre de cycles qui compte vraiment).
% - Leurs faire comprendre qu'en général c'est + facile de faire d'abord l'implémentation en floating point, puis passer step by step en fixed point et voir quel ajout de fonction faire foirer le calcul. Bcp de gain de temps pour debugguer
% - Qd on importe un dossier, toujours aller chercher le .ioc et regénérer le code



\section*{Introduction}

During last week session, you learned how to use the LimeSDR Mini board and GNU Radio to receive and demodulate in software packets transmitted by the MCU. Yet, at very high data rates, the software processing of an entire digital signal processing chain becomes inefficient as it requires increasing amounts of computing resources. In this hands-on, you will learn how the parts of the GNU Radio receiver (low-pass filtering and preamble detection) can be offloaded to a hardware accelerator. We will here focus on the detection of the preamble, as the low-pass filter is already implemented using an IP. The hardware acceleration further allows the software-defined radio to process data streams at higher data rates that are otherwise not reachable. Multiple tools used here need to be installed first. Please refer to the wiki of the Git. The FPGA design with the low-pass filter and the packet presence detection is available on the Git. Do not forget to synchronize it before starting the hands-on.

\section*{Objectives}
\begin{itemize}
    \item Become familiar with FSK modulation and non-coherent detection;
    \item Study the performance of a complete communication chain;
    \item Implement simple detection and synchronization algorithms;
    %\item Study the impact of synchronization issues.
\end{itemize}

\input{CMSIS}
\section{Fixed-point representation}
%
As a recall, when you are working on data with Python and Matlab, you generally deal with variables stored in the floating point representation as depicted in Fig.~(\ref{fig: 32}). This representation is close to what you learned as the \emph{scientific notation} and is convenient for simple operations as addition and multiplication.
%
\begin{figure}[H]
    \centering
    \includegraphics[width=\textwidth]{figs/Floating32.png}
    \caption{32-bit floating point example. The $1.0$ of $*1.25$ is implicit, meaning if all the fraction bits are $0$, we have $1.0$ by default}.
    \label{fig: 32}
\end{figure}
%
Unlike floating-point, the \emph{fixed-point} representation does not require 32 or 64 bits. This representation is closer to the binary format. The first bit is the \emph{sign} bit. Then we go from the Most Significant Bit (MSB) to the Least Significant Bit (LSB) by decreasing power of $2$. \\
Take the example of $8$-bit fixed-point. This means we can go in the range [$-128, 127$].
% A fixed-point representation does not give more than that, meaning we must store an implicit \emph{scaling factor} in memory to recover any initial value at the end. \\
% A way to see it is to always consider the signal is quantized in the range $\pm 1$ with a precision depending on the size of fixed-point representation. If you are measuring temperatures in the range $\pm 45$ [$\degree$C], this means you consider an implicit scaling factor of $45$ and you simply know:
% \begin{itemize}
%     \item $01111111 \rightarrow 1 \rightarrow 45$ [$\degree$C] is the maximum value.
%     \item $11111111 \rightarrow -1 \rightarrow -45$ [$\degree$C] is the minimum value (sign bit to 1)
%     \item $00000000 \rightarrow 0 \rightarrow 0$ [$\degree$C] is zero.
% \end{itemize}
%
\begin{bclogo}[couleur = gray!20, arrondi = 0.2, logo=\bcinfo]{Rescaling fixed points}
Think about multiplying your temperature signal by $8$, any value initially $>0.125$ will become larger than one and you will loose all the information about the signal. This is called \textbf{overflow}. In that case, you must first divide your signal by $8$ (equivalent to a $3$-bit shift to the right $\gg 3$).
% , and keep in mind your maximum value is now $45*8 = 360$ [$\degree$C].
This is called \textbf{rescaling} \\
\\
 In order to avoid overflows or precision loss, it often appears some functions rescale a signal in their computations. This information is given in the doc and you have to take it into account to recover the true amplitude of your final result. Note \emph{arm$\_$rfft$\_$q15} does rescale the signal, with a shift depending on the RFFT size, find the corresponding table in the doc.
\end{bclogo}
%
A practical example of an $8$-bit fixed-point signal is shown in Fig.~(\ref{fig: q4_4}). The default \textit{name} for such a signal is \emph{q1.7}. However, it is conventional to change this \textit{name} to keep track of the previous scalings. This is often critical when you convert a signal from fixed to float or inversely.
In this case, we have a \emph{q4.4} where we fixed the radix point.
% (keep in mind there is no practical difference).
%
\begin{figure}[H]
    \centering
    \includegraphics[width=10cm]{figs/fixed4_4.png}
    \caption{8-bit fixed point example \emph{q4.4}}
    \label{fig: q4_4}
\end{figure}
%
You can find a specific note of ARM on the fixed-point representation with pieces of code, named \emph{ARM$\_$fixed$\_$point}, in the \emph{Technical resources} directory on Moodle. \textbf{Read it carefully}, this is not wasted time. \\ \url{https://moodle.uclouvain.be/mod/folder/view.php?id=197742}. \\
Also find a description of the rules used by Matlab in practice for an FIR filter: \\ \begin{footnotesize}\url{https://fr.mathworks.com/help/dsp/ug/fixed-point-precision-rules-for-avoiding-overflow-in-fir-filters.html}\end{footnotesize}.

\section{Demodulating a capture with GNU Radio}

The teaching team recorded a capture containing five packets sent with the S2LP radio.
As a first introduction to GNU Radio, you will demodulate this capture using the 2-FSK demodulator and CFO estimators you wrote in the previous hands-on.

\subsection{Installing the \texttt{gr-fsk} module}

For this session, we provide you a version of the project that includes both a full software implementation for the MCU and a GNU Radio
module compatible with the S2LP radio. Download this project on Moodle and unzip it.

GNU Radio provides many built-in digital signal processing blocks (similarly to LabView), however these blocks are not sufficient to implement a functional FSK receiver.
The teaching team hence developed a new \href{https://wiki.gnuradio.org/index.php/OutOfTreeModules}{out-of-tree module} named
\texttt{gr-fsk} that contains all the blocks (preamble detection, synchronization, demodulation, packet parser, ...) needed to build an entire receiver chain for the S2LP radio.
Before running GNU Radio, this module must be compiled and installed. To this end, open a terminal, go into the \texttt{telecom/gr-fsk} folder and execute the following commands:
\begin{bash}
    mkdir build
    cd build/
    cmake ..
    sudo make install
\end{bash}
The module is now installed.

\subsection{Running the GNU Radio Companion}

GNU Radio may either be used directly using Python scripts or through a GUI software named GNU Radio Companion. It is much more convenient to use the latter to design and experiment receiver chains.
Start the GNU Radio Companion (\textit{``Programming -> GNU Radio Companion''}) and open the application \texttt{gr-fsk/apps/decode\_capture.grc}.
This application reads a prerecorded capture from a file
and demodulates it using the FSK receiver architecture that has been explained in the simulation hands-on. The operations carried out by the different blocks are exactly identical to those of the simulation framework we provided you.

\texttt{.grc} files are designs which can be graphically edited in the companion, but they cannot directly be executed.
Instead, the companion generates Python scripts based on the \texttt{.grc} design. To generate the corresponding script, press the second button starting from the left (i.e., the one with a box and an arrow) shown in Fig.~\ref{fig:buttons}.
The generated script can then be executed by pressing the \textit{Run} button (third button starting from the left).
\begin{figure}[H]
    \centering
    \includegraphics[scale=1]{figures/buttons.PNG}
    \caption{GNU Radio buttons to generate the Python script, running and stopping the application.}
    \label{fig:buttons}
\end{figure}

\begin{bclogo}[couleur = gray!20, arrondi = 0.2, logo=\bcinfo]{Setting the path of the \textit{File Source}}
    You may need to fix the path of capture in the block \textit{File Source}.
    The capture is located at \texttt{gr-fsk/misc/fsk\_capture.mat}.
    After modifying the design in the companion, re-generate the corresponding Python script before launching the application.
\end{bclogo}

When running the application, the console on the bottom left corner on the screen indicates the events processed by the different blocks.
Since both the estimation of the CFO and the demodulation functions are not implemented in this version of the project,
you should observe that 5 packets are detected with a CFO of \SI{0}{\hertz} and are incorrectly demodulated (all payload bytes are demodulated to zeroes).

\subsection{Modifying the \texttt{gr-fsk} module}

To correctly demodulate the capture, you now need to plug in the demodulator and CFO estimator from the previous hands-on into the \texttt{gr-fsk} module.
The \texttt{gr-fsk} blocks are written in Python and are all located in the folder \texttt{gr-fsk/python}.
Open the file \texttt{gr-fsk/python/demodulation.py}, which implements the \textit{Demodulation} block. For all \texttt{gr-fsk} blocks, the teaching team wrote the boilerplate code that
handles the buffers inside GNU Radio. On the contrary, the operations that are specific to the signal processing (e.g., demodulation, estimating the CFO or STO, ...) have been outsourced
to external functions. These functions have almost similar prototypes to the corresponding functions of the simulation framework.
\textbf{You may hence validate your signal processing functions in the simulation framework before inserting them in GNU Radio.}

Nonetheless, it is important that you gain some understanding of how custom GNU Radio blocks are written.
Please take some time to dive into \texttt{demodulation.py}. The functions \texttt{\_\_init\_\_} (initialization of the block with the specified parameters),
\texttt{forecast} (indicates to GNU Radio how many input samples are needed to provide $N$ output samples) and \texttt{general\_work} (actual function that performs the processing) are common to all GNU Radio blocks.
The companion however does not directly read the Python files to understand how a block is implemented. Instead, each block comes with a
YAML file, read by the companion, which includes all the information required by the tool: types of the input and output signals, parameters,
callback functions, ... Open the file \texttt{gr-fsk/grc/fsk\_demodulation.block.yml} and browse it also.

Once you went over the Python and YAML files, put the demodulation function you wrote for the previous hands-on in the external \texttt{demodulate} function. Repeat the same procedure for the CFO estimation in the function
\texttt{cfo\_estimation} of \texttt{gr-fsk/python/synchronization.py}. Then, before running the GNU Radio application again, you need to re-do an installation to propagate your changes:
\begin{bash}
    cd build/
    sudo make install
\end{bash}
\begin{bclogo}[couleur = gray!20, arrondi = 0.2, logo=\bcinfo]{Usage of the build folder}
    Creating and populating the build directory with \texttt{cmake} needs only to be done once.
    Afterwards it is sufficient to only do an install to propagate your changes in the Python files to the system.
\end{bclogo}

After the installation of your modifications, re-launch the application.
If the two functions \texttt{demodulate} and \texttt{cfo\_estimation} are correctly implemented, you should now observe that all 5 packets are rightly demodulated (with the payload bytes increasing succesively from 0 to 99)
and that the CFO of each packet is approximately located around \SI{7800}{\hertz}.

\section{Live demodulation with a cable}

Now that the \texttt{gr-fsk} blocks are fully functional, the next step consists in receiving and demodulating in real-time packets from the MCU.
The project contains a full MCU software implementation with a driver to interact with the S2LP radio.
This software features an evaluation mode of the radio that transmits several packets in a row with different transmit (Tx) output power levels.
To verify the entire chain (Tx and Rx) at a very high signal-to-noise ratio (SNR), we first use an SMA cable to connect the S2LP radio and the LimeSDR Mini.
The reference datasheet of the radio is available at the following link: \href{https://www.st.com/resource/en/datasheet/s2-lp.pdf}{\textcolor{blue}{[S2-LP datasheet]}}.

\subsection{Preparing the setup}

\begin{enumerate}
    \item Connect the S2LP radio on the MCU to the Rx port of the LimeSDR Mini board using the SMA cable.
    \item Connect the Nucleo board to your computer.
    \item Connect the LimeSDR Mini board to your computer \st{using the extension USB cable.
        \textbf{Never connect the LimeSDR Mini board directly to your computer, as interference from your computer may corrupt the received signal.}} \textcolor{red}{$\rightarrow$ due to a USB 2.0/3.0 compatibility issue, please do not use the extension cable. We will reach back to you when we have a solution.}
    \item If on VirtualBox, pass both devices to the guest system (\textit{"Devices -> USB -> ..."}). On WSL, see install guidelines how to attach the LimeSDR.
\end{enumerate}

\begin{bclogo}[couleur = gray!20, arrondi = 0.2, logo=\bcinfo]{Passing the LimeSDR Mini to the VM}
    Depending on your computer, you may have difficulties to pass the LimeSDR Mini to the guest system.
    When passing the devices in the VirtualBox menu, use the command \texttt{lsusb} in a terminal to verify if the device has correctly been passed.
    If you are unable to pass the board after a few trials, shutdown the VM, \textbf{close the VirtualBox manager and restart it}, and relaunch the VM.
\end{bclogo}

\subsection{Setting up the MCU}

Open the MCU software project \texttt{mcu} with Eclipse (\textit{"File -> Open projects from file system ..."}).
In the configuration file \texttt{Core/Inc/config.h}, ensure that the following macros are defined as
\begin{itemize}
    \item \texttt{ENABLE\_RADIO 1}: initialize the S2LP radio when booting.
    \item \texttt{RUN\_CONFIG EVAL\_RADIO}: use the radio evaluation mode.
    \item \texttt{ENABLE\_UART 1}: enable the UART.
    \item \texttt{DEBUGP 1}: enable the debug prints.
\end{itemize}

Beside the general configuration file, the parameters of the radio evaluation mode are defined in \texttt{Core/Inc/eval\_radio.h}.
Open this file and try to understand the different parameters.

Afterwards, compile the software and flash the Nucleo board. Open a serial console to \texttt{/dev/ttyACM0} and press the button B1 to start
transmitting packets.

\begin{bclogo}[couleur = gray!20, arrondi = 0.2, logo=\bcinfo]{MCU build error}
    If you encounter an error when building the MCU software project, this is probably because you need to update the project settings. To do so, please open the \texttt{ioc} and click \textit{Yes} when prompted to migrate the project to the new version. After, \textbf{do not forget} to re-generate. Finally, you should be able to compile the project. Some errors about \texttt{\_getpid} and \texttt{\_kill} may be left, but you can safely ignore them.
\end{bclogo}

\subsection{Running the GNU Radio application}

The teaching team prepared a GNU Radio application ready to receive packets from the S2LP radio.
Open the application \texttt{gr-fsk/apps/eval\_limesdr.grc}. Instead of reading a stream from a file, the application uses the LimeSDR Mini
to retrieve samples. Try to understand the effects of the following variables in the application:
\begin{itemize}
    \item \texttt{packet\_len}: length of the packet, in bytes.
    \item \texttt{rx\_gain}: gain of the amplification stage in the receiver chain, in decibels.
    \item \texttt{detect\_threshold}: threshold of the preamble detection. \\
    Open the Python file \texttt{gr-fsk/python/preamble\_detection.py} to understand its behavior.
    \item The cut-off frequency and transition width of the low pass filter.
\end{itemize}
If the role of some of these parameters are unclear to you, please ask the assistant for clarifications.

Finally, launch the GNU Radio application and then start transmitting packets using the MCU. You should be able to receive correctly all transmitted packets.
\begin{bclogo}[couleur = gray!20, arrondi = 0.2, logo=\bcinfo]{Help, my virtual machine becomes unstable or freezes!}
    If you observe that your virtual machine is unable to smoothly run the GNU Radio application, you may need to increase the number
    of CPU cores and the memory shared with the VM.
\end{bclogo}

\subsection{Optional: Recording a capture}

When debugging GNU Radio blocks, recording a capture is a very useful technique to run the same experiment again in identical conditions.
An application which performs a recording of the samples acquired by the LimeSDR Mini is provided in \texttt{gr-fsk/apps/record\_capture.grc}.
Try to use this application to record a new capture while the MCU transmits packets, and then demodulate it using \texttt{gr-fsk/apps/decode\_capture.grc}.

\section{Live demodulation over the air}

In the ecomonitoring application, the MCU needs to communicate with the gateway over the air, and not using a cable.
The final part of this hands-on session consists in using the antennas to transmit and receive the packets.

\subsection{Frequency allocation}

Since all wireless transmissions use the same medium (i.e., the air), it is important that each group uses a different carrier frequency to avoid interfering with other groups.
To this end, please use the following frequency allocation scheme, in which each group is allocated a bandwidth of \SI{2}{\mega\hertz}:
\begin{table}[h]
    \centering
    \begin{tabular}{c|c}
        Group number & Allocated carrier frequency\\
        \hline
        Group A & \SI{860}{\mega\hertz}\\
         Group B & \SI{862}{\mega\hertz}\\
         Group C & \SI{864}{\mega\hertz}\\
         ... & ...
    \end{tabular}
    \caption{Frequency allocation scheme among the groups.}
    \label{tab:freq_alloc}
\end{table}

The allocated carrier frequency must be configured both in the MCU and in the GNU Radio Companion.
\begin{itemize}
    \item In the MCU code, open the file \texttt{Core/Inc/s2lp.h} which contains the parameters used by the radio.
    Try to understand the different parameters and modify the macro \texttt{BASE\_FREQ} to your allocated carrier frequency.
    \item In GNU Radio, modify the variable \texttt{carrier\_freq}.
\end{itemize}
For more details on the different parameters and modes in which the S2LP radio can be configured to, have a look at Tables 12 (page 13) and 40 (page 28) in the S2-LP datasheet.

\subsection{Running the GNU Radio application}

Once the carriers are correctly configured, replace the SMA cable with the antennas and run the application.
Since transmitting over the air implies that the signal at the receiver will have a much weaker power compared to a transmission over a cable,
it is necessary to increase the Rx gain in the LimeSDR Mini. This can be done using the GUI when the application is running. \textbf{Run the application and set the gain at \SI{60}{\decibel}.}
You should now be able to receive packets over the air.
You might need to try several distances (between 1 and 5 meters) and different values of the variable \texttt{detect\_threshold} to achieve a functional communication.

\subsection{Measuring the noise level}

When receiving a packet, the receiver estimates its signal-to-noise ratio (SNR), defined as
\begin{equation}
    \textrm{SNR} = \frac{P}{\sigma^2},
\end{equation}
with $P$ being the power of the received signal and $\sigma^2$ is the variance of the additive white gaussian noise (AWGN).
Estimating the SNR is required to experimentally evaluate the performance of the receiver.
The theoretical explanation of the estimation of both random variables has been provided in the appendix of the previous hands-on.
In short, the power of the received signal is dynamically estimated using the preamble of the packet.
On the contrary, the power of the noise is a fixed value which has to be estimated before the application is launched.

In practice, an estimate of the noise power $\sigma^2$ can be obtained using the application \\
\texttt{gr-fsk/apps/estimate\_noise.grc} \textbf{when the antenna is connected to the LimeSDR Mini}.
The applications prints each second an estimate of $\sigma^2$.
Since the noise level depends on the Rx gain, it is important that this gain is configured to the same value as the one used in the application
\texttt{gr-fsk/apps/eval\_limesdr.grc}.

\begin{enumerate}
    \item Estimate the noise power using the application with a Rx gain of \SI{60}{\decibel}.
    \item Repeat the operation by bypassing the low-pass filter (right-click on the block to enable/bypass it).
    What is the effect of the low-pass filter on the noise power?
    \item Copy the estimated value (in linear scale, not in decibels)
    with low-pass filter enabled to the the variable \texttt{estimated\_noise\_power} of the \texttt{gr-fsk/apps/eval\_limesdr.grc} application.
    \item Regenerate and launch the application, and start transmitting packets at increasing distances.
    At which SNR levels does the radio start failing to demodulate packets?
\end{enumerate}

\section{Demo D5: double buffering acquisition and classification}
%
During this LELEC2102 project, we expect you to be able to present quick demos on specific topics, as scheduled in the program for this semester. They can be seen as small checkpoints to make sure you master the important basic blocks of the project. These demos should take you less time than writing a detailed report while still providing a good occasion to develop new skills, learn by doing and finally get some feedback along the way! There is no need to write anything for a demo but you must make sure it will run "live" smoothly. \\
\\
The \textbf{demo D5 on Friday December 1$^{\text{st}}$ 4.15 PM} will consist in checking you master the provided code and you will be able to modify it accordingly for some improvements during the second semester. For this demo on Friday, we expect you to:
%
\begin{itemize}
    \item Show an acquisition of a full feature vector from a single button press.
    \item Show a feature vector-long melspectrogram acquired with the jack cable, then with the microphone.
    \item Show a classification result obtained live from acquisition with your MCU.
\end{itemize}

% Report guidelines are moved to Moodle only.
% This is used as food for thoughts as there is no longer a R6

% \section{Report R6}

% The report R6 (due on \textbf{Nov 22, Monday 6.30 PM}). We expect you to:

\begin{bclogo}[couleur = gray!20, arrondi = 0.2, logo=\bcquestion]{Optimization levels trade-off}
    \begin{itemize}
        \item What are the number of cycles taken by the MAC function
            per authenticated byte for different message lengths ? Example parameters : 1 byte up to 200 bytes with steps of 1 byte for the different compiler optimization levels:
            \texttt{-O0},
            \texttt{-O1},
            \texttt{-O2},
            \texttt{-O3},
            \texttt{-Os}.\footnote{%
                To understand what these optimization levels do (as well as
                \texttt{-Og}), see
                \url{https://gcc.gnu.org/onlinedocs/gcc/Optimize-Options.html}.
                The optimizations controlled by these flags happen at the compiler
                level, and each file is compiled individually.

                Since optimization is based on gathering information on the
                code and how it is used, this limited (file-level) view may lead to
                missed optimization opportunities.
                Link-time optimization (LTO) is a technique that enables
                project-level view for optimizations (see
                \url{https://gcc.gnu.org/onlinedocs/gcc/Optimize-Options.html})
                and that you might want to use at some stage of your project.
                To add such custom compiler and linker flags, use the
                \textbf{Miscellaneous} section of the Compiler/Linker settings.
            }
        \item What is the code size of the MAC function when compiled with the different optimization levels ?
    \end{itemize}
\end{bclogo}



\end{document}

%%%%%%%%%%%%%%%%%%%%%%%%%%%%%%%%%%%%%%%%%%%%%%%%%%%%%%%%%%%%%%%%%%%%%%%%%%%%
