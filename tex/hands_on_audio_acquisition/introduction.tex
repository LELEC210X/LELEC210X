\section*{Introduction}

In this Hands-On, you will be introduced to the basics of audio acquisition in an embedded context. This Hands-On is building up on the skills you developed during the H2a. Therefore, make sure you master the concepts, tools \textit{and} practical skills mentioned in that Hands-On. Moreover, make sure you have carefully read and understood the \texttt{Guide - how to use the boards.pdf} in the \texttt{Technical ressources} folder on Moodle. As for the H2a, we will provide you with a functional baseline code to start with for this Hands-On. Then, you will be guided step by step to develop the key skills needed for this part of the project.

\begin{bclogo}[couleur = gray!20, arrondi = 0.2, logo=\bcinfo]{Explanation of the hands-on boxes}
In this note, there are a few boxes presenting additional information:
\begin{itemize}
    \item \bcinfo will provide you some more detailed explanations.
    \item \bcattention will explain typical mistakes that might lead to errors or a non functional system.
    \item \bcquestion will provide you with additional questions or experiments that will improve your understanding of the system. We advise you to leave them for the end of the hands-on as they are not critical.
\end{itemize}
\end{bclogo}
