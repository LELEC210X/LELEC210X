% arara: pdflatex: { interaction : nonstopmode }
% arara: biber: { options: ['--isbn-normalise'] }
\documentclass[t, aspectratio=169, english, table]{_style/tudelft-beamer}
% \includeonlyframes{current}% beamer documentation \S 4.3.3

\title[Git Tutorial]{How to use Git for the LELEC210X project}
\subtitle{About how Git can be your best friend}
\author[Assistants Team\inst{1}]{The LELEC210X Assistants\inst{1}.}
\date{29-01-2024}
% define a graphic to be shown next to the title
\titlegraphic{\includegraphics[width=0.35\paperwidth]{figures/git.pdf}}

% optional:
% \setlength{\titlesplitpos}{-0.5\paperwidth}
% mind that in this case, the white pane is the background, and the blue one on top.

\institute[Universities of Somewhere and Elsewhere]{
  \strut\llap{\inst{1}\,}UCLouvain}
\date[LELEC210X]{Project in Electrical Engineering, LELEC210X}

\newcommand{\absimage}[4][0.5,0.5]{%
	\begin{textblock}{#3}%width
		[#1]% alignment anchor within image (centered by default)
		(#2)% position on the page (origin is top left)
		\includegraphics[width=#3\paperwidth]{#4}%
\end{textblock}}


\begin{document}


\titleframe


\begin{frame}[fragile]{Welcome!} % some commands, e.g. \verb require [fragile]
  \frametitle{TU Delft presentation template}
  \framesubtitle{In \LaTeX\ using the package Beamer}
  
  This template can be used to make a presentation in the 2022 version of the TU Delft style described here:
  \url{https://www.tudelft.nl/huisstijl/middelen/presentaties}
  
  The icons have been converted to pdf, so they can be included crisply against a colored background:
  
\end{frame}


\begin{frame}
  A digital version of this presentation can be found at \url{https://gitlab.com/novanext/tudelft-beamer}.

  Here's a QR code made by \LaTeX, pointing to the same link:

  \begin{center}  
    \qrcode{https://gitlab.com/novanext/tudelft-beamer}
  \end{center}

  Slides like these are straightforward to make, the following contains more fancy examples. Using all of these slide options in one presentation is probably too much for your audience\ldots
\end{frame}


{
  \renewcommand{\bginsert}{\hfill\raisebox{0.45\paperheight}{\color{tud purple}%
    \clap{\includegraphics[width=1.1\paperwidth]{uclouvain-logo.pdf}}}\hfill}
  \setbeamercolor{background canvas}{bg=tud pink}
  \renewcommand{\sectionsubtitle}{Hope you will be inspired!}
  \section{Random Subset\\of Features}
}

\end{document}
