\section{Telecommunication}
\subsection{Packet Error Rate}
To obtain the experimental packet error rate (PER) as a function of the signal-to-noise ratio (SNR), one must measure the number of packets incorrectly demodulated by GNU Radio at several SNRs. To perform the characterization, we will use the \textit{eval\_limesdr\_fpga} flowgraph. To minimize its computational load, you should first deactivate the \textit{QT GUI Sink} block by clicking on it and pressing \textbf{D} (you can reactivate it by pressing \textbf{E}). Do not forget to regenerate the flow graph afterwards (Cube icon).

You can then launch the GNU Radio app either through the GUI, or using the command \textit{python eval\_limesdr\_fpga.py} in the \textit{gr-fsk/apps} folder. Useful measurements data will be written to the file \textit{measurements.txt}, which should be located either in the \textit{apps} folder, or in the folder from which you launched the python command. Be careful, this file will be overwritten each time you launch the flowgraph.

When running the flowgraph, you will be able to specify manually the transmit power used by the MCU. Do so before pressing the B1 button on the MCU to start the transmissions.  This is needed so that GNU Radio will be able to print packets information and the corresponding SNR, but it has no effect on the transmission itself. As the gain is increased after pressing the button, you need to update the \textit{tx\_power} for every batch of packets that is sent. Do not forget to set the gain and the threshold factor before making the noise estimation, and only then enable the detection.

Once all packets have been transmitted, you can post-process the \texttt{measurements.txt} file using the following Python script:
    \begin{center}
    \texttt{uv run read-measurements measurements.txt}\\
    \end{center}
The script extracts different metrics for each of the packet received, successfully or not. You can then use the output dataframe as you wish in order to create your PER-SNR curves. It is not mandatory to use this script if you prefer to start from scratch.

When doing the measurements, please take into account the following remarks:
\begin{itemize}
    \item \textbf{Physical connection:} Use the SMA cable for your measurement to minimize external perturbations. Avoid placing the SMA cable too close to an electronic device or cable that might generate interference. Moreover, to avoid any issue, plug in your computer to a proper power supply to avoid the resource usage limitations that can occur when running on battery. Finally, make sure the different SMA connections (cable, attenuators) are screwed tightly.
    \item \textbf{Attenuators:} As we want to study the communication under an AWGN noise, we will need to use a high RX gain, otherwise the noise is truncated on too few bits at the ADC output, making it not white. Try to use a 70dB RX gain in the flowgraph. To avoid the saturation while using a 70db RX gain, you should try to have 70 dB of attenuation. You have received a 30 dB attenuator from the teaching assistant and will receive an additional one of 10 dB. For the 30 dB remaining, you can borrow it to another group. Plan your experimentations before to avoid borrowing it for too long, and keep track of where your attenuator is.
    \item \textbf{MCU:} set the MCU in radio evaluation mode. Do not forget to increase the number of transmitted packets at each Tx power level, and to decrease the waiting time between each packet transmission\footnote{You can decrease this value in the \textit{eval\_radio.c} file by modifying the HAL delays in the last for loop. For example, set the delays to 10 ms (instead of 200 ms), so that one packet is approximately sent every 20 ms.}. Use a range of Tx powers (between -30 and 15 dBm) that will enable you to reach useful SNRs. Remember that every time the B1 button is pressed, the MCU radio increases the Tx power by 1~dBm. \footnote{You can also change this power step by modifying the for loop in the {eval\_radio.c} file.}
    \item \textbf{GNU Radio:} setup the RX gain to 70dB, modify the threshold factor to ensure the proper detection of lower power packets, and do not forget to perform a noise estimation. To avoid any computational load issue, deactivate the \textit{QT GUI Sink} and avoid performing other tasks on your computer during your measurement. \footnote{If you find GNURadio output too much data on the command prompt, you can go in the corresponding python block and comment the concerned logging. Do not forget to recompile the \textit{gr-fsk} package.} It can happen that the setup starts to continuously demodulate packets incorrectly. In this case, stop the simulation. In the measurement file or at the \textit{read\_measurement.py} output, you can discard unrelevant data and aggregate multiple results.
    %\item \textbf{GNU Radio:} to avoid any unwanted impact of the PPD, we are going disable it for this characterization. To do so, open the LimeSDR FPGA Source block in the \textit{eval\_limesdr\_fpga.grc} and go in the last tab, \textit{"Custom FPGA"}. There, set the enable field to \textit{No}. As we still need to have some sort of preamble detection, right click on the preamble detector and select enable. Right click on the Flag detector and  select disable.
    %\item \textbf{FPGA:} program the FPGA with the bitstream you obtained during the H4. If you do not have a functional bistream, you can use the following one, {LimeSDR-Mini\_lms7\_trx\_HW\_1.2\_auto.rpd}, provided in the FPGA folder of the repository. This is the default one of the LimeSDR-Mini and it does not include the FIR. Do not use it if you a functional bitstream.
    %program the FPGA with the bitstream \textit{LimeSDR-Mini\_lms7\_lelec210x\_HW\_1.0\_auto.rpd} provided in FPGA folder of the repository. \textit{\textcolor{blue}{This bitstream is different from the one  used in the H4. It uses a 31-taps lowpass filter with a cut-off frequency at $\frac{f_s}{4}$, where $f_s$ is the sampling rate.}}
\end{itemize}

One key point when comparing measurement and simulation results is to use similar parameters. You have already characterized the transfer function of the FIR implemented in the FPGA, do not forget to use a similar one in the simulation setup. Moreover, ensure you use the same parameters for the payload length, the oversampling factor, and for the Moose Algorithm. \\

The simulation graphs of PER-SNR have two axis: the $SNR_o$ on the decision variable, and the $SNR_e$ on the output of the low-pass filter. The last is the one you have access to in the GNU Radio flowgraph and you should use for your comparison.

%\subsection{Carrier frequency offset estimation}
%You can also extract a histogram of the estimated frequency offsets using the Python script \\ \emph{read\_file\_for\_CFO\_SNR.py} with one of the output files generated for the PER-SNR curve. Ensure that the LimeSDR had enough time to warm up, and use the measurement with the largest SNR.



%%%%%%%%%%%%%%%%%%%%%%%%%%%%%%%%%%%%%%%%%%%%%%%%%%%%%%%%%%%%%%%%%%%%%%%%%%%%%%%%%%%%%%%%%%
\newpage
\begin{comment}
\appendix
\section{Questions}
\subsection{Packet error rate}
Obtain the packet error rate curves as a function of the SNR using the simulation framework. Compare the PER-SNR curve obtained assuming perfect synchronization with the one using all synchronization algorithms. Identify and comment the differences, highlighting their origins with relevant graphs and explanations. Determine the bottleneck in the communication chain. Furthermore, compare these simulated curves with the PER-SNR curve obtained in practice, superimposed on the simulation results. Explain and motivate the differences.

\subsection{Carrier frequency offset estimation}
Discuss the Moose algorithm, based on simulation results and practical measurements. Give and motivate a method to choose the length $N$ of the blocks used by Moose.

Regarding the measurements, obtain a histogram of the carrier frequency offset. What is the shape of the histogram and how can it be linked to $N$?

\subsection{SNR-distance measurements}
Another important aspect of your communication chain is the communication distance that can be achieved. Previously, you identified the required SNR to ensure a desired BER or PER. The next step is therefore to determine the variations of the SNR with the communication distance.

To do so, take some measurements of the estimated SNR (provided in GnuRadio) for several communication distances. Use the antennas with fixed TX and RX gains\footnote{Suggestion: use a RX gain above 50 dB and a TX gain larger than 0 dBm.} and set the preamble detection threshold accordingly. Make sure that the packets are correctly demodulated and average the estimated SNRs over several packets. \textbf{We recommend you to do the measurements outside or in an open area.}


\section{Characterization report}
\textcolor{red}{Add questions in question list, only keep pratical measurement hints.}

\subsection{Packet error rate}
Obtain the packet error rate curves as a function of the SNR using the simulation framework. Compare the PER-SNR curve obtained assuming perfect synchronization with the one using all synchronization algorithms. Identify and comment the differences, highlighting their origins with relevant graphs and explanations. Determine the bottleneck in the communication chain.

Furthermore, compare these simulated curves with the PER-SNR curve obtained in practice, superimposed on the simulation results. Explain and motivate the differences.

\paragraph{Practical measurements} \textcolor{red}{give these details in a separated document?}
To obtain the PER-SNR curve in practice, one must measure the number of packets correctly demodulated at several SNRs. We provide you with a script that can parse the console output of GnuRadio to extract the number of received packets, the number of packet errors, the average estimated SNR as well as the measured CFO.

To use it, you should remove all GUI from GnuRadio (by clicking on the Options block and setting the generate options to 'NO GUI', after having disabled all GUI blocks) and generate the Python script. Then, run the following command in a console:
    \begin{center}
    \textit{python eval\_limesdr\_fpga.py | tee output.txt}\\
    \end{center}
This will launch your Python script and save all console outputs in the file \textit{output.txt}. After the LimeSDR initialisation, you can start sending packet using the MCU radio. At the end, click on enter.

To post-process the \textit{output.txt} file, use:
    \begin{center}
    \textit{python read\_file\_for\_CFO\_SNR.py output.txt}\\
    \end{center}
This should provide all the information required to obtain one point on the PER-SNR curve.

Please take into account the following remarks:
\begin{itemize}
    \item Use the SMA cable with three 30 dB attenuators. \textcolor{red}{how to rent?}
    \item \textbf{FPGA:} program the FPGA with \textcolor{red}{???}
    \item \textbf{MCU:} set the MCU in radio evaluation mode. Do not forget to increase the number of sent packets, as well as decrease the time between each packet sent\footnote{You can decrease this value in the \textit{eval\_radio.c} file by modifying the HAL delays in the last for loop. For example, set the delays to 10 ms (instead of 500 ms), so that one packet is approximately sent every 20 ms. In this case, we measured that sending 1000 packets should take roughly 80 seconds.}. Use a range of TX power that enable you to target the desired SNRs. Remember that every time the B1 button is pressed, the MCU radio will increase the TX power by 1 dBm.
    \item \textbf{GnuRadio:} do not forget to first measure the estimated noise power, with the chosen RX gain. Set the preamble detection threshold to a value that allows you to sense packets at small SNRs. However, a too small detection threshold will be triggered by the noise.
    \textit{Hint: the teaching team used a RX gain of 70 dB with a detection threshold close to 0.005, varying with increasing TX power.}
    \item \textbf{GnuRadio: Disable the hardware preamble detector}, in the LimeSDR FPGA Source block.
    \item For the comparison with the simulated PER curve (with the same modulation parameters, synchronization algorithms and lowpass filter band), remember that the SNR estimated in practice is the SNR right after the lowpass filter and not the one on the decision variable. \textcolor{red}{+updated sim files}
\end{itemize}






\subsection{Carrier frequency offset estimation}
Discuss the Moose algorithm, based on simulation results and practical measurements. Give and motivate a method to choose the length $N$ of the blocks used by Moose.

Regarding the measurements, obtain a histogram of the carrier frequency offset. What is the shape of the histogram and how can it be linked to $N$?

\textit{Hint: you can easily extract all estimated frequency offset and a raw histogram using the Python file \emph{read\_file\_for\_CFO\_SNR.py} with one of the output files generated for the PER-SNR curve. Ideally, take the one with the largest SNR and/or the one where the MCU radio and the LimeSDR had enough time to warm up.}


\subsection{SNR-distance measurements}
Another important aspect of your communication chain is the communication distance that can be achieved. Previously, you identified the required SNR to ensure a desired BER or PER. The next step is therefore to determine the variations of the SNR with the communication distance.

To do so, take some measurements of the estimated SNR (provided in GnuRadio) for several communication distances. Use the antennas with fixed TX and RX gains\footnote{Suggestion: use a RX gain above 50 dB and a TX gain larger than 0 dBm.} and set the preamble detection threshold accordingly. Make sure that the packets are correctly demodulated and average the estimated SNRs over several packets. \textbf{We recommend you to do the measurements outside or in an open area.}


\textcolor{red}{Delete section about path loss exponent. Students can determine it based on their measurements if they want.}

Based on your measurements, determine the path loss exponent. Indeed, assuming the power at the receiver is given by $P_R\:=\:\frac{P_T~G_T~G_R}{L}$, with $P_T$ the power at the transmitter, $G_T$ the gain of the emitter antenna and $G_R$ the gain of the receiver antenna, the path loss $L$ can be extracted\footnote{See LELEC2795 - Communication systems.}. The model for $L$ (in dB) to use is:
\begin{equation*}
    L_{[dB]}\:=\:A \:+\:10~n~\log_{10}(r),
\end{equation*}
with $r$ the communication distance (in meters), $A$ a constant and $n$ the path loss exponent to determine. By expressing the received SNR based on $P_R$, a linear regression can be used to extract the path loss exponent and a constant gathering $P_T$, $G_T$, $G_R$ and $A$.


\newpage
\section{Oral exam questions}

\subsection{BER-PER-SNR}
\begin{itemize}
    \item What is the difference between the input SNR and the SNR on the decision variable? And the SNR after the lowpass filter?
    \item What is a BER? And a PER? What is the link between these two quantities?
    \item How do you measure a PER in practice? Is it possible to measure it if the packet payload is unknown?
\end{itemize}


\subsection{CFO estimation}
\begin{itemize}
    \item Is there a maximal CFO that can be corrected with the Moose algorithm? Where is it coming from? How can it be increased?
\end{itemize}



\newpage
\appendix
\section{Appendix - details PER curve}
\paragraph{Practical measurements}
To obtain the PER-SNR curve in practice, one must measure the number of packets correctly demodulated at several SNRs. Please follow the steps detailed below:

\begin{enumerate}
    \item Use the SMA cable with three 30 dB attenuators, plugged between the MCU radio and the LimeSDR. \textcolor{red}{+ add details about how to rent attenuators...}
    \item \textbf{FPGA:} Program the LimeSDR FPGA with the lowpass filter having 31 taps, that you designed during the Hands-on session 5. \textcolor{red}{NOT SURE ABOUT FILTER!!}
    \item \textbf{MCU:} Program the MCU in radio evaluation mode (by changing \textit{RUN\_CONFIG} in \textit{config.h}). In the \textit{eval\_radio.h} header, make sure to set the minimum TX power level to -15 dBm and the maximum level to -5 dBm. The number of packets should also be increased to 1000. In this case, the MCU radio will send 1000 packets with a payload of 100 bytes, starting at the minimum TX power level. Then, every time you press the B1 button, the power level is increased by 1 dBm and 1000 new packets are sent. By default, packets will be sent every 1 second. Luckily, you can decrease this value in the \textit{eval\_radio.c} file by modifying the HAL delays in the last for loop. For example, set the delays to 10 ms (instead of 500 ms), so that one packet is approximately sent every 20 ms. In this case, we measured that sending 1000 packets should take roughly 80 seconds. \textbf{However, if you notice that your GnuRadio cannot keep demodulating the packets at this speed, you can increase the delays between the packets. You should avoid decreasing the number of sent packets as this will impact the accuracy of the measured packet error rate.}
    \item \textbf{GnuRadio:} The application \textit{eval\_limesdr\_fpga.grc} should be used. However, some changes must be made:
    \begin{itemize}
        %\item First, make sure that all demodulation parameters are the same as in the simulation framework. Namely, you probably changed the length of the sequence used in the Moose algorithm during the H4b session. You should put it back to $N=4$. However, as you noticed, you may not be able to correct any frequency offset in this case, depending on your devices. This is the reason why we suggest you to also perform a fixed frequency offset correction, by slightly modifying the \textit{carrier\_freq} variable in GnuRadio. We encourage you to explain and motivate this fixed frequency offset correction in the question about the CFO histogram. Before going any further, make sure that your communication chain is still working in ideal conditions (large SNR) with this correction. \textcolor{red}{not needed anymore, can keep N=2. Do not hint about fixed CFO then.}
        \item Open \textit{estimate\_noise\_fpga.grc}. Set the RX gain to 70 dB and estimate the noise power with the SMA cable and the 30 dB attenuators plugged-in. (\textit{Hint: it should be close to 1e-6}). Encode this estimated noise power value in \textit{eval\_limesdr\_fpga.grc}.
        \item In \textit{eval\_limesdr\_fpga.grc}, set the RX gain to 70 dB and the preamble detection threshold to 0.003. Moreover, in the LimeSDR FPGA Source block, \textbf{disable the hardware preamble detector}. Make a sanity check by setting the TX MCU radio noise power to -10 dBm and monitoring the number of received packets. You should normally receive all sent packets, with approximately 5 extra packets. These extra packets are received during the initialisation of the LimeSDR because the preamble detection threshold is low. Keep in mind that you will need to discard them later when you will compute the resulting packet error rate.
        \item When the \textit{eval\_limesdr\_fpga.grc} has been parameterized properly, we suggest you to generate a Python file without Graphical User Interface (GUI). This is possible by disabling the QT GUI Sink and QT Gui Range blocks (click on 'd' after having selected the blocks). Then, enable the disabled variable blocks (by clicking on 'e'). Finally, click on the Options block and set the generate options to 'NO GUI'. Generate the Python script (named \textit{eval\_limesdr\_fpga.py}).
    \end{itemize}
    \item Open a terminal and go to the gr-fsk/apps folder. Run the Python script \textit{eval\_limesdr\_fpga.py} with the following command:\\
    \begin{center}
    \textit{python eval\_limesdr\_fpga.py | tee output\_min\_15dBm.txt}\\
    \end{center}
    This will make sure that all the console output of the GnuRadio Python script will be saved in the file \textit{output\_min\_15dBm.txt}. When the LimeSDR has been intialized, click on the B1 button and start sending packets.

    \item When all packets has been received (the blue LED on the MCU is not blinking anymore), click on enter. You now have a file containing the results of all the packets sent at the TX power level of -15 dBm. This file is post-processed using:\\
    \begin{center}
    \textit{python read\_file\_for\_CFO\_SNR.py output\_min\_15dBm.txt}\\
    \end{center}
    This will extract for you the CFO histogram as well as the average SNR and the number of packets received and demodulated properly. You can then deduce the packet error rate at this SNR.

    \item Redo the steps 5 to 6 for another SNR. Do not forget to change the filename and to first launch the \textit{eval\_limesdr\_fpga.py} script, wait for the LimeSDR initialisation and then click on the B1 button.

    \item Compare the practical PER-SNR curve with the simulated one (\textbf{with the same modulation parameters, synchronization algorithms and lowpass filter band}). Remember that the SNR estimated in practice is the input SNR and not the one on the decision variable.
\end{enumerate}

\end{comment}
