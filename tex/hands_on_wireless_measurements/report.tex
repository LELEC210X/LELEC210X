\section{Report and demonstration}

\subsection{Demonstration}

On the next session, we expect from each group a quick demonstration showing a correct reception of 5 packets transmitted over the air using the radio evaluation mode
at the default Tx power level of \SI{-16}{\decibel\meter}. The noise power should be estimated and set in the application such that the SNR estimation is correct.

\subsection{Report}

Please upload a report that contains contents for the two wireless communications sessions.
The guidelines for the previous session are to be found in the corresponding hands-on.

For this session, we expect a \textbf{maximum} 2 pages with
\begin{itemize}
    \item Screenshot of the GNU Radio Companion console showing the correct CFO estimation and demodulation of \textbf{one} packet from the capture we provided you.
    \item Values of the noise power $\sigma^2$ in the four following cases:
        1) a \SI{60}{\decibel} Rx gain with low-pass filter (LPF) 2) a \SI{60}{\decibel} Rx gain with the LPF bypassed 3) a \SI{70}{\decibel} Rx gain with LPF 4) a \SI{70}{\decibel} Rx gain with the LPF bypassed.
    Discuss and explain how the Rx gain and the presence (or absence) of the LPF impact the noise power.
\item Using over the air transmissions with a \SI{60}{\decibel} Rx gain and the LPF enabled, evaluate the minimum SNR required to correctly receive packets.
    Does this SNR value correspond to the simulation results you observed in the simulation? Discuss.
\end{itemize}
