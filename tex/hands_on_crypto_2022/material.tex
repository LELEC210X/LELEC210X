\section*{Material \& preparation}

\begin{comment}[couleur = gray!20, arrondi = 0.2, logo=\bcinfo]{}
\vspace{0.2cm}
\end{comment}
For this hands-on session, you will need:
\begin{itemize}
    \item The Nucleo MCU board with its sensing and power-management board ;
    \item A USB cable to connect your computer to the Nucleo.
    \item A working Virtual Machine (VM) with
        \begin{itemize}
            \item the MCU project from H3b loaded on the MCU,
            \item the following python packages installed\footnote{%
                    Install them with \texttt{sudo pip install package-name}.
                }:
                \texttt{cryptography}, \texttt{pyserial},
            \item the H\handsOnN{} script (\texttt{packet\_inspector.py}).
        \end{itemize}
        You should be able to run the H\handsOnN{} code package:
        program the MCU, then, in the directory containing the
        H\handsOnN{} code, run \texttt{python packet\_inspector.py -{}-input
        /dev/ttyACM0} (change \texttt{ttyACM0} to something else if needed).
        If the MCU is in acquisition mode (blue LED not blinking), you should
        see that that packets are received and decoded.
\end{itemize}
